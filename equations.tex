\documentclass[12pt,a5 paper]{article}
\usepackage{amsmath}
\usepackage{cclicenses}
\title{Program on Writing Equations}
\author{SONAM BHARTI \\ Nalanda College of Engineering,\\ CHANDI \\ \byncsa}
\date{28 June 2019}

\begin{document}
\maketitle
\newpage
We will demonstrate the creation of equations with some samples.  Let
us start with the model of an inverted pendulum: 


\begin{align}
\frac d{dt}
\begin{bmatrix} x_1\\x_2\\x_3\\x_4  \end{bmatrix} &  =
\begin{bmatrix} 
0 & 0 & 1 & 1\\
0 & 1 & 0& 1\\
0 & -\gamma & 0 & 0\\
0 & \alpha & 0 & 0\\
\end{bmatrix}
\begin{bmatrix}x_1 \\x_2\\x_3\\x_4\end{bmatrix} +
\begin{bmatrix}0\\0\\-\delta\\-\beta \end{bmatrix}
\delta \mu\\
\intertext{Proportional, Integral, derivative controller is most popular in industry.
 It has three tuning parameters: $K$, $\tau_1$ and $\tau_d$}
 u(t) & = K\left( e(t) + \frac 1 {\tau_1}
\int_0^t e(t)dt +
 \tau_d \frac {de(t)} {dt}\right )\\ 
%
\intertext{Let us go through the discrete equivalent of this equations :}
u(n) - u(n-1)  & = K[e(n)-e(n-1) + \frac {T_s} {2\tau_i} \{e(n)+e(n-1)\} \nonumber \\
& + \frac {\tau_a} {T_s} \{e(n) - 2e(n-1)+e(n-2)\}]
\end{align}
\end{document}





\section{this is the second section}                                                                                                    //distribute the section

\intertext{Let us go through the discrete equivalent of this equations :}\\                                            //insert text in between the equations
u(n) - u(n-1)  & = K[e(n)-e(n-1) + \frac {T_s} {2\tau_i} \{e(n)+e(n-1)\} \nonumber \\                       //nonumber is used either to ignore the numbering when equation is in 2 part or while we don't have the need for numbering
& + \frac {\tau_a} {T_s} \{e(n) - 2e(n-1)+e(n-2)\}]


\section{This is the fist section}
\label{sec:100}


\label{eq:GFG}\\ \\

We will now discretise the PID controller given in equation \ref{eq:GFG}.
Section \ref{sec:100} shows how to write equations.\\